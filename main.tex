\documentclass{article}
\usepackage[utf8]{inputenc}
\usepackage{a4wide}
\usepackage{amsmath}
\usepackage{amssymb}
\usepackage{fancyhdr}
\usepackage{lastpage}
\usepackage{graphicx}
\usepackage[danish]{babel}
\usepackage{float}


\title{
    10022 Fysik 1 - eksperiment 5 \\
    \large{Undersøgelse af rullende stift legeme}
}
\author{Freja Ø. Strandlod (s214730) \and Hugo M. Nielsen (s214734) \and Mikael H. Hoffmann (s214753)}
\date{\today}

\pagestyle{fancy}
\fancyhf{}
\rhead{Fysik 1 - eksperiment 4}
\lhead{Freja Ø. Strandlod, \and Hugo M. Nielsen, \and Mikael H. Hoffmann}
\rfoot{\thepage{} af \pageref{LastPage}}


\begin{document}
\maketitle

\section*{Formål \& hypotese}
Formålet med dette forsøg er at bestemme intertimomentet af et stift legeme. Legemet er her en \emph{iOLab} indsat i to taperuller, så den kan rotere. Inertimomentet skal først bestemmes vha. energibetragtninger og senere impulsmomentsætningen. 

\section*{Metode}
En \emph{iOLab} indsættes i to taperuller, så den kan rotere om sin egen y-akse. Den placeres på et skråt bræt 13.8 cm over bordpladen. Herefter slippes den og triller ned af skråplanet. Forsøget gentages 10 gange for at formindske usikkerheden. 

Først bestemmes intertimomentet vha energibetragtninger. Energibevarelse siger, at:

\begin{align*}
    U_1+K_1=U_2+K_2
\end{align*}

Vi sætter bordpladen til $h=0$ og \emph{iOLab'en} starter i hvile. Dette reducerer til ligningen:

\begin{align*}
    U_1=K_2
\end{align*}

Den kinetiske energi i slutsituationen er givet ved summen af den rotationelle energi og den translationale energi. Dermed ser ligningen således ud:

begin{align*}
    U_1=K_rot
\end{align*}

Nu har vi inertimomentet for hele det stive legeme, men det er kun inertimomentet for \emph{iOLab'en}, som skal findes. Derfor skal de to taperullers inertimoment trækkes fra. 

\section*{Resultater}


\subsection*{Resultatbehandling}


\subsection*{Fejlkilder}
Vores primære fejlkilde er usikkerheder på vores måledata. Vi har gentaget forsøget for at minimere disse, men en væsentlig begrænsning er længden af brættet. Jo længere skråplanet er, jo længere distance bliver målingerne taget over og bliver derfor mere præcise. 

\section*{Konklusion}


\pagebreak

\appendix
\section{Bilag}

\end{document}






